\documentclass[twocolumn]{IEEEtran}

\begin{document}
\title{Evolutionary Aproach to Traveling Salesman Problem}
\author{Angel David Corredor}
\date{}
\maketitle

\begin{abstract}
    The paper presents a technique for solve the Traveling Salesman Problem using a modified Genetic Algorithm.
    Then the experiments will be compared with traditional GA in some problems of TSPLibrary.     
\end{abstract}

\section{Introduction}

The travelling salesman problem is one of the classic NP-hard optimization problems.
In this problem, there are $n$ cities and the distances between the cities are given by a 
distance matrix $D=d_{ij}$ ($d_{ij}$ the distance between city $i$ and $j$).
There is a salesman who must visit each city exactly once and minimize the time of the complete tour.
It is assumed that the speed of the salesman is constant ($v_c$) and he tries to minimize
the time of the tour. 
The objective function is given by:

\begin{equation}
    f(\bar{x}) =
    \sum_{i=1}^{n-1} (t_{x_i,x_{i+1}})
    + t_{x_n,x_1},
    \bar{x}=(x_1,...,x_n)
\end{equation}

where $\bar{x}$ represents a tour, which contains all of the cities exactly once, 
$t_{x_i,x_{i+1}}$ is the time of the travel between $x_i$ and $x_{i+1}$ and it is calculated by:

\begin{equation}
    t_{x_i,x_{i+1}} =
    \frac{d_{x_i,x_{i+1}}}{v_c}
\end{equation}

Clearly, $f$ is the total time of the tour. The aim is to find $x$ which minimizes $f$. 

\section{INdividual representation}

The encoding of each individual is a permutation of $n$ elements called tour,
this represents the order in which the traveler visit the cities.

\section{Conclutions}

%\begin{thebibliography}{X}
    %Main https://cs.adelaide.edu.au/~zbyszek/Papers/TTP.pdf (este da una proposicion mejor de multiobjetivo)
    
    %%Instancias del problema https://cs.adelaide.edu.au/~optlog/CEC2014COMP_InstancesNew/
    %%Benchmark https://cs.adelaide.edu.au/~ec/research/ttp/2014gecco-ttp.pdf
    %%BO-TTP https://cs.adelaide.edu.au/~markus/pub/2018gecco-emottp.pdf
    %%Mas cosas https://cs.adelaide.edu.au/~ec/research/combinatorial.php   
%\end{thebibliography}

\end{document}