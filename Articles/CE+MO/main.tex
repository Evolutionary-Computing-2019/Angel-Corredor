\documentclass[twocolumn]{IEEEtran}

\begin{document}
\title{Coevolutionary Aproach to Biobjective Traveling Thief Problem}
\author{Angel David Corredor}
\date{}
\maketitle

\begin{abstract}
    The paper presents a technique for solve the Biobjective Traveling Thief Problem using a Coevolutionary approach.
    There are 2 different populations that resolves separately the subproblems of TTP but work together
    to solve BO-TTP using the friend concept. Finally experiments will be presented using TTP benchmark.  
\end{abstract}

\section{Introduction}

The problem is a combination of two well-known optimization problems (TSP and KP).
Two models of TTP are introduced and their characteristics are investigated. 
These two models are different from each other based on the way in which the sub-problems are
interdependent. It is shown that solving each sub-problem in isolation is not effective and the 
two sub-problems have to be considered together.

\subsection{Traveling Salesman Problem}

The travelling salesman problem is one of the classic NP-hard optimization problems.
In this problem, there are $n$ cities and the distances between the cities are given by a 
distance matrix $D=d_{ij}$ ($d_{ij}$ the distance between city $i$ and $j$).
There is a salesman who must visit each city exactly once and minimize the time of the complete tour.
It is assumed that the speed of the salesman is constant ($v_c$) and he tries to minimize
the time of the tour. 
The objective function is given by:

\begin{equation}
    f(\bar{x}) =
    \sum_{i=1}^{n-1} (t_{x_i,x_{i+1}})
    + t_{x_n,x_1},
    \bar{x}=(x_1,...,x_n)
\end{equation}

where $\bar{x}$ represents a tour, which contains all of the cities exactly once, 
$t_{x_i,x_{i+1}}$ is the time of the travel between $x_i$ and $x_{i+1}$ and it is calculated by:

\begin{equation}
    t_{x_i,x_{i+1}} =
    \frac{d_{x_i,x_{i+1}}}{v_c}
\end{equation}

Clearly, $f$ is the total time of the tour. The aim is to find $x$ which minimizes $f$. 

\subsection{Knapsack Problem}

The knapsack problem is another NP-hard optimization problem. 
In this problem, there are $m$ items $I_1, ..., I_m$, each of
which has a value ($p_i$) and a weight ($w_i$).
Also, there is a thief who wants to fill his knapsack by taking some of these items.
The capacity of the knapsack is limited ($W$).
The aim of the thief is to pick the items to maximize their total value while
their total weight does not exceed the knapsack capacity.
The problem is modelled as follows:

\begin{equation}
    maximize \; g(\bar{y}) =
    \sum_{i=1}^m p_iy_i, 
    \bar{y} = (y_1, ..., y_m)
\end{equation}

\begin{equation}
    subject \; to
    \sum_{i=1}^m w_iy_i \leq W
\end{equation}

where $y_i \in \{0,1\}$ and $y_i = 1$ indicates picking item $i$ and
$g(y)$ is the total value of the picked items. The aim is to
maximize $g$ in a way that the total weight of the items does not
exceed the capacity of the knapsack. 

\subsection{Treveling Thief Problem}
In $TTP_2$  there are two objectives, namely maximizing the
total value while minimizing the travel time. However, three
new parameters are added to interconnect two sub-problems:

\begin{itemize}
    \item The speed of the travel ($v_c$ in (2)) is related to the
    knapsack’s current weight, 
    ($v_c = (v_{max} - W_c \frac{v_{max}-v_{min}}{W})$)
    where $W_c$ is the current weight of knapsack, $v_{max}$ and $v_{min}$
    are the maximum and minimum velocity of the thief, and
    $W$ is the maximum capacity of knapsack. Note that the
    time of the travel between two cities is calculated via (2).
    According to this formulation, the speed of the thief ($v_c$)
    decreases when the weight of the knapsack increases. Note
    that the thief travels in $v_{max}$ when the knapsack is empty
    ($W_c=0$) and in $v_min$ when the knapsack is full ($W_c=W$).

    The thief has rented the knapsack and he has to pay the rent. 
    The knapsack’s rent rate is $\$R$ per time unit.
    The aim of $TTP_2$ is to maximize the following function:

    \begin{equation}
        G(x,z) = g(z) - R*f(x,z)
    \end{equation}

    where $g$ is the total value of all picked items, $R$ is the rent
    per time unit, and $f$ is the total time of the tour. $x$ and $z$ are the
    tour and the picking plan (the solutions of TSP and KP subproblems), respectively.
    Note that the travel time and weight of the picked items have been connected through the changes
    of the speed of travel. Thus, the function $f$ needs to get both $x$
    and $z$ as inputs to calculate the total time. Also, the total value
    of the picked items has been connected to the travel time by
    the rent rate.

    \item The value of the picked items drops by time.
    In fact, the final value of the item at the end of the travel is not the
    same as its value when the thief picked the item. This
    value is dependent on travel time. The dropping rate is
    defined by $Dr^{\lceil \frac{T_i}{C} \rceil}$ where $T_i$ is the total time that the item i
    is carried by the thief and C is a constant. The new value
    of the item i at the end of the tour is calculated by 
    $p_i*Dr^{\lceil \frac{T_i}{C} \rceil}$ where $i$ is the item number. \\
\end{itemize}

Note that the rule is to pick the items before the tour is
completed. Hence, the thief cannot pick any more items when
he gets back in the starting city. The aim of $TTP_2$ is to satisfy
the following objectives:

\begin{equation}
    G(x,z) =
\left\{
	\begin{array}{ll}
		min & f(x,z)  \\
		max & g(x,z)
	\end{array}
\right.
\end{equation}

where $x$ and $z$ are the tour and picking plan, respectively. \\

The function $f$ is the time of the tour which is calculated
according to the distance of the cities and the weight of the
picked items. Also, $g$ is the total value of all picked items after
completing the tour (note that the value of the items drops by
the time that they were in the knapsack). The function $f$ is
related to the knapsack weight, in such a way that the heavier
the knapsack gets, the slower the thief can travel, resulting in
greater time being taken to complete the tour. Also, the value
of the picked items after completing the tour depends on the
time of the tour and the original value of the items. The aim is
to maximize the total value of the picked items and minimize
the time for completing the tour. 

\section{Problem Encoding}

There are 2 independant populations solving each problem independently but linked with a friend in the other population. %%Pendiente citacion

\subsection{TSP population}
The encoding of each individual is a permutation of $n$ elements called tour,
this represents the order in which the traveler visit the cities. 
Aditionaly each individual have a number that identify his friend in the KP population.

\subsection{KP population}
The encoding of each individual is a bit-array with size $m$ that represents the desition of picking each item.
In adition have a number that identify his friend in the TSP population (Encoded in base-2).

\subsection{Friends}
...

\section{Conclutions}

%\begin{thebibliography}{X}
    %Main https://cs.adelaide.edu.au/~zbyszek/Papers/TTP.pdf (este da una proposicion mejor de multiobjetivo)
    
    %%Instancias del problema https://cs.adelaide.edu.au/~optlog/CEC2014COMP_InstancesNew/
    %%Benchmark https://cs.adelaide.edu.au/~ec/research/ttp/2014gecco-ttp.pdf
    %%BO-TTP https://cs.adelaide.edu.au/~markus/pub/2018gecco-emottp.pdf
    %%Mas cosas https://cs.adelaide.edu.au/~ec/research/combinatorial.php   
%\end{thebibliography}

\end{document}